\chapter{Алгоритмы машинного и глубокого обучения в задаче о кредитном скоринге}
       	
	\section{Предобработка данных кредитной истории клиентов банка}
	В данной работе используются обезличенные данные АО «Альфа-Банка»\footnote{\cite{Alfabank} (дата обращения 25.11.2025)}. Данные состоят из 12 файлов (\texttt{train\_data\_0.pq - train\_data\_11.pq}), содержащих информацию о платежах клиентов банка. В каждом из 12 файлов содержится информация о 250 000 клиентах. При этом один клиент может иметь несколько кредитов, и каждому такому клиенту соответствует персональный \texttt{id} (идентификационный номер). Отдельно имеется файл \texttt{train\_target.csv}, который состоит из 3 млн строк, и каждая строка соответствует клиенту с меткой (флагом) равной 0 (отсутствие дефолта) или 1 (наличие дефолта). 
	Задача предобработки данных состоит в структурировании исходной информации, т.е. формирование единого датасета, выделение важных признаков (колонок), выявление аномальных клиентов (определение аномальности  будет приведено ниже), визуализация данных и тестирование моделей МО и ГО на этих данных. Программный код формирования единого датасета реализован в листинге \ref{train_data_csv_all_py} (см. приложение 1). 
{\co Пояснение к листингу: \ref{train_data_csv_all_py}} 
	\begin{enumerate}
	       \item Строки 1 -- 3. Импортируются необходимые библиотеки: \texttt{pandas} -- для работы с табличными данными, \texttt{os} -- для работы с файловой системой и \texttt{pyarrow.parquet} -- для чтения файлов формата \texttt{.parquet}.
               \item Строка 5. Задается путь \texttt{path = "train\_data"} к папке,
в которой находятся исходные файлы формата \texttt{.pq}.
               \item Строки 6 -- 13. Запускается цикл \texttt{for}, который перебирает все файлы в папке \texttt{train\_data}. Формируется имя поочередного файла \texttt{train\_data\_i.pq}, создается объект \texttt{ParquetDataset} для текущего файла, из которого данные считываются в \texttt{DataFrame (df)}. Затем выполняется агрегация данных по признаку \texttt{id}, вычисляются средние значения признаков, после чего данные сохраняются в соответствующий \texttt{csv} -- файл в папку \texttt{train\_data\_csv\_all}.
               \item Строка 14. Выводится список файлов каталога \texttt{train\_data} для проверки корректности формирования файлов.
               \item Строки 16 -- 21. Задается путь к папке с полученными \texttt{csv}-файлами \texttt{train\_data\_csv\_all}. Создается пустой список \texttt{frames} для последующего хранения отдельных \texttt{DataFrame}. Затем запускается цикл по файлам в папке \texttt{train\_data\_csv\_all}, который последовательно перебирает все файлы в формате \texttt{.csv}. Результат сохраняется в \texttt{DataFrame (df)} и добавляется в список \texttt{frames}. 
               \item Строки 23 -- 26. Все элементы списка \texttt{frames} объединяются в единый \texttt{DataFrame result} с помощью функции \texttt{pd.concat}. Полученный датасет сохраняется в файл \texttt{1\_data\_csv\_all.csv}, после чего заново считывается в переменную \texttt{df\_all} для последующего анализа.
        \end{enumerate}
        
        После преобразования получился следующий датасет. В таблице \ref{df_fragment} приводится фрагмент датасета:
        
\begin{table}[H]
\centering
\caption{Фрагмент преобразованного датасета, содержащий первых 5 клиентов}
\label{df_fragment}
\begin{tabular}{|c|c|c|c|c|c|c|}
  \hline
  \rowcolor[gray]{.9}
   id & enc\_paym\_0 & enc\_paym\_1 & enc\_paym\_2 & enc\_paym\_3 & enc\_paym\_4 & flag \\
  \hline
  1750000 & 0.17 & 0.17 & 0.17 & 0.33 & 0.67 & 0 \\ \hline
  1750001 & 0.00 & 0.75 & 0.75 & 0.75 & 0.75 & 0 \\ \hline
  1750002 & 0.39 & 0.33 & 0.72 & 0.67 & 1.17 & 0 \\ \hline
  1750003 & 0.18 & 0.23 & 0.41 & 0.50 & 0.55 & 0 \\ \hline
  1750004 & 0.60 & 0.60 & 0.60 & 0.60 & 1.20 & 0 \\ \hline
\end{tabular}

\vspace{3mm} 

\begin{minipage}{0.98\textwidth}
  \fontsize{9}{11}\selectfont
  \justifying
Примечание: id -- идентификатор заявки; \texttt{enc\_paym\_0}, \dots, \texttt{enc\_paym\_n} -- статусы ежемесячных платежей за последние $n$ месяцев; flag -- статус кредита (0=кредит полностью оплачен). Полный датасет состоит из 61 признака, включая дополнительные характеристики по кредитам.
\end{minipage}
\end{table}
        
{\co Пояснение к агрегированию клиентов в листинге: \ref{train_data_csv_all_py}}    
     
        Агрегация клиентов по идентификатору \texttt{id} в строке 11 необходима для определения итоговой метки (флага) каждого клиента, поскольку одному заемщику может соответствовать несколько кредитных договоров. Задача состоит в присвоении каждому кредиту одного заемщика единую итоговую. метку. В данном исследовании в качестве общего значения для всех кредитов используется среднее исходных значений. Для понимания идеи приводится следующий пример в виде таблицы  \ref{table_1_1:dfgr} :
        
\begin{center}
  \begin{longtable}{|c|c|c|c|c|c|c|c|c|}
   \caption{Дисциплина оплаты кредитов клиента (id = 1)}
    \label{table_1_1:dfgr} \\
    \hline
    \rowcolor[gray]{.9}
     id  &  N  &  M1 &  M2 & M3 & M4 & M5 & M6 & flag   \\
     \hline
     1   &  1   &  1   &  0    &  1   &  1   &  0   &  1  &\multirow{3}{*}{0}\\
     \hhline{|--------~|} % горизонтальная линия с 1-ой строки по по 8-ую строку
     1   &  2   &  0   &  \cellcolor[gray]{.9}\textbf{1}  &  0   &  1   &  1   &  2  &\\
     \cline{1-8}
     1   &  3   &  1   &  2    &  0   &  0   &  3   &  2  &\\
     \cline{1-9} 
     & &\multicolumn{6}{|c|}{ Среднее значение } & \\
     \hline
     1  &   &  0.67  &  1  &  0.33  &  1  &  1.33  &  1.67  & 0 \\
     \hline
\end{longtable}
    
\begin{minipage}{\textwidth}
   \fontsize{9}{11}\selectfont
   \justifying
   Источник: составлено автором на основе: Соревнование на данных кредитных историй [Электронный ресурс] / Open Data Science. – URL: \url{https://ods.ai/competitions/dl-fintech-bki} (дата обращения: 25.11.2025).
\end{minipage}
\end{center}
        
        В данной таблице признаки означают следующее:
\begin{enumerate}
   \item \texttt{id} -- идентификационный номер клиента;
   \item \texttt{N} -- номер кредита;
   \item \texttt{M1,...,M6} -- статусы погашения в течение 6 месяцев (0=платеж вовремя оплачен, 1=задержка платежа на 1 день, 2=задержка платежа на 2 дня, 3=задержка платежа на 3 дня);
    \item \texttt{flag} -- статус кредита (0=кредит полностью оплачен).
\end{enumerate}
        
        В таблице \ref{table_1_1:dfgr} на пересечении N=2 (второго кредита) и M2 (платеж во втором месяце) выделена цифра 1, означающая, что платеж по второму кредиту во втором месяце был оплачен с опозданием на 1 день.
        
        Таким образом, данные трех строк, соответствующих трем кредитам клиента, в таблице \ref{table_1_1:dfgr} были усреднены и преобразованы в одну строку, которая содержит агрегированную информацию по клиенту с \texttt{id = 1}. В целом агрегировать клиентов можно не только по среднему значению, но и по моде или медиане. Однако в данной работе выбрано среднее значение, поскольку оно обладает важными статистическими свойствами, такими как несмещенность и состоятельность. После группирования данных был сформирован новый датасет, состоящий из 3 млн клиентов, каждому из которых соответствует одна итоговая метка. Получившийся датасет содержит 61 признак, из которых далее необходимо отобрать наиболее информативные, т.е. такие признаки, существенно влияющие на точность алгоритмов МО и ГО.
        
        
\subsection{Метод главных компонент}

Метод главных компонент (англ. $principal\ component\ analysis$, $PCA$) -- метод сокращения размерности данных, позволяющий уменьшать количество признаков с сохранением максимального объема исходной информации\footnote{\cite{PCA} (дата обращения: 25.11.2025).}, на которых обучаются модели МО и ГО.

Как уже было отмечено выше сформированный датасет содержит 61 признак (столбец). Задача состоит в том, чтобы найти такие признаки, на которые модели МО и ГО показывали приемлемую точность. Следует отметить, что редукция признаков может уменьшить точность алгоритмов, поэтому необходимо внимательно следить за процессом сокращения данных. В качестве тестового алгоритма был выбран алгоритм $Random\ Forest$ (случайный лес), поскольку в статье С.В. Смирнова\footnote{\cite{Smirnov} (дата обращения: 25.11.2025).}  был проведен анализ предпочтения исследователей к алгоритмам МО, показавший, что наиболее популярным является случайный лес.

Из преобразованного датасета изначально выбираются только некатегориальные признаки и формируется новый датасет, содержащий 41 признак. Ниже приводится смысл этих признаков:

\begin{enumerate}
    \item \texttt{pre\_pterm} -- плановое количество дней с даты открытия кредита до даты его закрытия;
    \item \texttt{pre\_fterm} -- фактическое количество дней с даты открытия кредита до даты его закрытия;
    \item \texttt{pre\_loans\_next\_pay\_summ} -- сумма следующего платежа по кредиту;
    \item \texttt{pre\_loans\_outstanding} -- оставшаяся невыплаченная сумма кредита;
    \item \texttt{pre\_loans\_total\_overdue} -- текущая просроченная задолженность по кредиту;
    \item \texttt{pre\_loans\_max\_overdue\_sum} -- максимальная просроченная задолженность по кредиту за весь срок;
    \item \texttt{pre\_loans\_credit\_cost\_rate} -- полная стоимость кредита;
    \item \texttt{is\_zero\_loans5} -- флаг: нет просрочек до 5 дней;
    \item \texttt{is\_zero\_loans530} -- флаг: нет просрочек от 5 до 30 дней;
    \item \texttt{is\_zero\_loans3060} -- флаг: нет просрочек от 30 до 60 дней;
    \item \texttt{is\_zero\_loans6090} -- флаг: нет просрочек от 60 до 90 дней;
    \item \texttt{is\_zero\_loans90} -- флаг: нет просрочек более чем на 90 дней;
    \item \texttt{pre\_util} -- отношение оставшейся невыплаченной суммы кредита к кредитному лимиту;
    \item \texttt{pre\_maxover2limit} -- отношение максимальной просроченной задолженности к кредитному лимиту;
    \item \texttt{is\_zero\_util} – флаг: отношение оставшейся невыплаченной суммы кредита к кредитному лимиту равно 0;
    \item \texttt{is\_zero\_over2limit} -- флаг: отношение текущей просроченной задолженности к кредитному лимиту равно 0;
   \item \texttt{enc\_paym\_0}, \dots, \texttt{enc\_paym\_n} -- статусы ежемесячных платежей за последние $n$ месяцев.
\end{enumerate}

Далее из этих 41 признака необходимо выбрать только такие, которые вносят наибольший вклад в информативность данных. В листинге  \ref{feat_41_py} (см. Приложение 1) реализован метод главных компонент. На таком наборе данных алгоритм показывает следующие метрики (см. Таблицу \ref{metrics_41}):

\begin{center}
  \begin{longtable}{|c|c|c|c|c|c|c|}
    \caption{Метрики точности на 41 признаке}
    \label{metrics_41} \\
    \hline
    \multicolumn{7}{|c|}{Алгоритм -- Случайный лес}\\
    \hline
    \rowcolor[gray]{.9}
    \specialcell{Flag\\метка} &
    \specialcell{Precision\\точность} &
    \specialcell{Recall\\полнота} &
    \specialcell{f1-score\\f1-мера} &
    \specialcell{Accuracy\\точность}  &
    ROC AUC &
    Количество    \\
    \hline
    \multicolumn{7}{|c|}{Тренировочная выборка}\\
    \hline
    0 & 1.00 & 0.97 & 0.98 &
    \multirow{2}{*}{0.97} &   
    \multirow{2}{*}{0.51} &  
    724588 \\                    
    \cline{1-4} \cline{7-7}
    1 & 0.00 & 0.00 & 0.00 &  
      &      & 25412 \\       
    \hline
    \multicolumn{7}{|c|}{Тестовая выборка} \\
    \hline
    0 & 1.00 & 0.97 & 0.98 &
    \multirow{2}{*}{0.97} &
    \multirow{2}{*}{0.50} &
    241569 \\
    \cline{1-4} \cline{7-7}
    1 & 0.00 & 0.00 & 0.00 & & & 8431 \\
    \hline
  \end{longtable}
  \begin{minipage}{\textwidth}
    \fontsize{9}{11}\selectfont
    \justifying
    Источник: составлено автором на основе: Соревнование на данных кредитных историй [Электронный ресурс] / Open Data Science. -- URL: \url{https://ods.ai/competitions/dl-fintech-bki} (дата обращения: 25.11.2025).
  \end{minipage}
\end{center}

Были получены результаты $PCA$:
\begin{equation}
%\setlength{\arraycolsep}{1.2ex} % компактные столбцы
\label{pca_lambda41}
\begin{alignedat}{5} % 5 лямбда в строке
\lambda_{1}^{(p)}  &= 22.1, & \lambda_{2}^{(p)}  &= 20.3, & \lambda_{3}^{(p)}  &= 17.9, & \lambda_{4}^{(p)}  &= 10.1, & \lambda_{5}^{(p)}  &=  9.7,\\
\lambda_{6}^{(p)}  &=  7.7, & \lambda_{7}^{(p)}  &=  4.4, & \lambda_{8}^{(p)}  &=  1.7, & \lambda_{9}^{(p)}  &=  1.5, & \lambda_{10}^{(p)} &=  0.9,\\
\lambda_{11}^{(p)} &= 0.6, & \lambda_{12}^{(p)} &= 0.5, & \lambda_{13}^{(p)} &= 0.4, & \lambda_{14}^{(p)} &= 0.3, & \lambda_{15}^{(p)} &= 0.2,\\
\lambda_{16}^{(p)} &= 0.2, & \lambda_{17}^{(p)} &= 0.2, & \lambda_{18}^{(p)} &= 0.1, & \lambda_{19}^{(p)} &= 0.1, & \lambda_{20}^{(p)} &= 0.1,\\
\lambda_{21}^{(p)} &= 0.1, & \lambda_{22}^{(p)} &= 0.1, & \lambda_{23}^{(p)} &= 0.1, & \lambda_{24}^{(p)} &= 0.1, & \lambda_{25}^{(p)} &= 0.1,\\
\lambda_{26}^{(p)} &= 0.1, & \lambda_{27}^{(p)} &= 0.1, & \lambda_{28}^{(p)} &= 0.1, & \lambda_{29}^{(p)} &= 0.1, & \lambda_{30}^{(p)} &= 0.1,\\
\lambda_{31}^{(p)} &= 0.0, & \lambda_{32}^{(p)} &= 0.0, & \lambda_{33}^{(p)} &= 0.0, & \lambda_{34}^{(p)} &= 0.0, & \lambda_{35}^{(p)} &= 0.0,\\
\lambda_{36}^{(p)} &= 0.0, & \lambda_{37}^{(p)} &= 0.0, & \lambda_{38}^{(p)} &= 0.0, & \lambda_{39}^{(p)} &= 0.0, & \lambda_{40}^{(p)} &= 0.0,\\
\lambda_{41}^{(p)} &= 0.0
\end{alignedat}
\end{equation}

В формуле \eqref{pca_lambda41} $\lambda_{1}^{(p)},\dots, \lambda_{41}^{(p)}$ -- это доли собственных чисел ковариационной матрицы, выраженные в процентах. Начиная с  $\lambda_{18}^{(p)}$ эта доля не превышает 0.2\%. Это означает, что существует 24 компонента, которые вносят незначительный вклад в информативность данных. Метод $PCA$ не представляет возможности точно определять какие именно признаки вносят существенный вклад в информативность данных, поэтому необходимо самостоятельно выбирать и удалять признаки с низким вкладом. Было сделано предположение, что наименее информативными признаками являются: \texttt{pre\_pterm}, \texttt{pre\_fterm}, \texttt{pre\_loans\_next\_pay\_summ}, \texttt{pre\_loans\_outstanding}, \texttt{pre\_loans\_total\_over} \\ \texttt{due}, \texttt{pre\_loans\_max\_overdue\_sum}, \texttt{pre\_loans\_credit\_cost\_rate}, \texttt{is\_zero\_loans5}, 
\texttt{is\_} \texttt{zero\_loans530}, \texttt{is\_zero\_loans3060}, \texttt{is\_zero\_loans6090}, \texttt{is\_zero\_loans90}, \texttt{pre\_util}, \texttt{pre\_maxover2limit}, \texttt{is\_zero\_util},  \texttt{is\_zero\_over2limit}. 
 
 Для подтверждения необходимо повторно проделать метод $PCA$ без 16 признаков, касательно которых было сделано предположение. В листинге \ref{feat_25_py} (см. приложение 1) реализуется метод главных компонент. 

\begin{enumerate}
   \item Строки 1--2. Задаётся файл с датасетом из 25 признаков, который считывается в объект \texttt{DataFrame(df)}. 
   \item Строки 4--10. Формируется список со статусами ежемесячных платежей \texttt{enc\_paym\_k}, $k = 0,\dots,24$.
   \item Строка 12. \texttt{X\_pay = df.loc[:, columns\_pay].copy()} -- из исходного датасета \texttt{df} выбираются 25 признаков, которые формируют матрицу \texttt{X\_pay}, содержащую информацию о платёжной дисциплине.
   \item Строка 13. Создаётся объект метода главных компонент PCA с параметрами по умолчанию.
   \item Строка 14. На матрице \texttt{X\_pay} обучается модель PCA.
   \item Строка 16. Вычисляются доли объясненной дисперсии для каждого главного компонента.
\end{enumerate}

Получается следующая значимость признаков, процентно выраженная в собственных числах ковариационной матрицы  \texttt{X\_pay}.

\begin{equation}
\label{pca_lambda41_modified}
\begin{alignedat}{5} % 5 лямбда в строке
\lambda_{1}^{(p)}  &= 65.5, & \lambda_{2}^{(p)}  &= 16.0, & \lambda_{3}^{(p)}  &= 5.7, & \lambda_{4}^{(p)}  &= 3.0, & \lambda_{5}^{(p)}  &= 1.9,\\
\lambda_{6}^{(p)}  &= 1.3,  & \lambda_{7}^{(p)}  &= 1.0,  & \lambda_{8}^{(p)}  &= 0.8, & \lambda_{9}^{(p)}  &= 0.7,  & \lambda_{10}^{(p)} &= 0.6,\\
\lambda_{11}^{(p)} &= 0.5, & \lambda_{12}^{(p)} &= 0.4, & \lambda_{13}^{(p)} &= 0.4, & \lambda_{14}^{(p)} &= 0.3,  & \lambda_{15}^{(p)} &= 0.3,\\
\lambda_{16}^{(p)} &= 0.3, & \lambda_{17}^{(p)} &= 0.3, & \lambda_{18}^{(p)} &= 0.2, & \lambda_{19}^{(p)} &= 0.2,  & \lambda_{20}^{(p)} &= 0.2,\\
\lambda_{21}^{(p)} &= 0.1, & \lambda_{22}^{(p)} &= 0.1, & \lambda_{23}^{(p)} &= 0.1, & \lambda_{24}^{(p)} &= 0.1,  & \lambda_{25}^{(p)} &= 0.1.
\end{alignedat}
\end{equation}

Из формулы \eqref{pca_lambda41_modified} можно заметить, что осталось только 25 признаков, где доли собственных чисел $\lambda_{i}^{(p)}$ не равны нулю, т.е. остались те признаки, которые вносят существенный вклад в информативность данных. Однако из таблицы \ref{metrics_41} видно, что некоторые метрики для алгоритма случайного леса равны нулю, что является неприемлемым для прогнозирования кредитных рисков. Например, это метрики $Recall$ и $Precision$, для дефолтных клиентов. На первый взгляд метод $PCA$ не оказал влияния на некоторые метрики алгоритма, поэтому требуется дальнейший детальный анализ.

\subsection{Определение аномального клиента}
\label{subsec:anom}

Рисунок \ref{fig:labels_before_anom} показывает, что доля недефолтных клиентов существенно велика и не соответствует статистике в реальной жизни. По данным Первого кредитного бюро около 20\%\footnote{\cite{Tengrinews} (дата обращения: 25.11.2025).}
казахстанских заемщиков имеют дефолтную кредитную историю, что подтверждает данное утверждение и указывает на наличие аномальности в классе недефолтных клиентов. 

\begin{figure}[H]
\centering
\includegraphics[scale=0.5]{./ch2/graphics/labels_before_anom.png}
\caption{Перекос в сторону недефолтных клиентов}
\label{fig:labels_before_anom}
\vspace{10pt}
\begin{minipage}{\textwidth}
        \fontsize{9}{11}\selectfont
        \justifying
        Источник: составлено автором на основе: Соревнование на данных кредитных историй [Электронный ресурс] / Open Data Science. -- URL: \url{https://ods.ai/competitions/dl-fintech-bki} (дата обращения: 25.11.2025).
    \end{minipage}
\end{figure}

В данной работе под аномальным клиентом понимается нетипичное поведение для большинства заемщиков. Речь идет о ситуации, когда клиент фактически не располагает достаточными денежными средствами для обслуживания кредита, однако продолжает вносить ежемесячные платежи с небольшими, но регулярными опозданиями, перезанимая необходимые суммы из других источников. В таблице \ref{table_1_1:dfgr} приведен пример неестественных выплат клиента, из которого видно, что платежная дисциплина клиента по отдельному кредиту не выглядит дефолтной, однако в среднем этот клиент не выплачивал обязательства пунктуально ни в один расчетный месяц (средние значения по месяцам представлены в последней строке таблицы). Ниже будет дано определение и алгоритм выявления таких клиентов в датасете. 

{\df Аномальным клиентом является такой клиент, для которого выполняются два условия:
\begin{equation}
\left\{
\begin{aligned}
\text{(a)}\;& \text{метка (флаг)} = 0, \\
\text{(b)}\;& \min(enc\_paym\_n) > 0,\quad n = 0,\ldots,24.
\end{aligned}
\right.
\tag{$anom$}
\label{anom}
\end{equation}
}

В формуле \eqref{anom} условие $(a)$ означает, что аномальный клиент в датасете является недефолтным, а условие $(b)$ -- по всем месяцам оплата кредитных обязательств проводилась с опозданием. В листинге \ref{anom0in1_py} (см. приложение 1) указана реализация формулы \ref{anom}. 

{\co Пояснение к листингу \ref{anom0in1_py}:} 
	\begin{enumerate}
		\item Строка 1. Читаем файл \texttt{train\_target.csv} с целевой переменной в датафрейме \texttt{df\_y}. 
		\item Строки 2--3. Из столбца \texttt{flag} датафрейма \texttt{df\_y} извлекаются метки для первых 1000000 клиентов, где \texttt{y} -- это исходные метки дефолта/недефолта, а \texttt{y\_} -- новые метки, которые будут изменяться в процессе нахождения аномальных клиентов. 
		\item Строка 5. Инициализируются счетчики \texttt{counter\_norm} -- число не аномальных клиентов, а \texttt{counter\_anom} -- число аномальных клиентов.
		\item Строка 6. Создаются пустые списки, в которые будут записываться индексы не аномальных (\texttt{ind\_norm}) и аномальных клиентов (\texttt{ind\_anom}).
		\item Строка 7. Запускается цикл по всем строкам матрицы \texttt{X\_pay}, где \texttt{i} это порядковый номер клиента (индекс строки), а \texttt{x\_pay} -- массив платежной истории по всем месяцам.
		\item Строки 8--11. Проверяется условие, которое определяет изначально недефолтных клиентов (\texttt{y\_[i] == 0}), у которых по всем месяцам наблюдаются задержки платежей (\texttt{min(x\_pay) > 0}). Если оба условия для клиента выполняются, то счетчик аномальных клиентов увеличивается на 1 (\texttt{counter\_anom +=1}) и пополняется список индексов аномальных клиентов (\texttt{ind\_anom.append(i)}), а его новой метке \texttt{y\_[i]} присваивается значение 1. Таким образом, часть клиентов, имевших исходную метку 0, переводится в класс аномальных клиентов с меткой 1. 
		\item Строки 12--14. В случае неудовлетворения этим двум условиям цикл относит клиента к не аномальным (\texttt{counter\_norm += 1}) и добавляет его индекс в список не аномальных клиентов (\texttt{ind\_norm}).
	\end{enumerate}
	
Данный алгоритм показал, что доля дефолтных клиентов увеличилась с 3.38\% до 35.45\% (см. Рисунок \ref{fig:labels_after_anom}). Это означает, что часть недефолтных клиентов перешла в класс дефолтных, и именно они  удовлетворяют определению \eqref{anom}.

\begin{figure}[H]
\centering
\includegraphics[scale=0.5]{./ch2/graphics/labels_after_anom.png}
\caption{Переход клиентов из класса «0» в класс «1»}
\label{fig:labels_after_anom}
\vspace{10pt}
\begin{minipage}{\textwidth}
        \fontsize{9}{11}\selectfont
        \justifying
        Источник: составлено автором на основе: Соревнование на данных кредитных историй [Электронный ресурс] / Open Data Science. -- URL: \url{https://ods.ai/competitions/dl-fintech-bki} (дата обращения: 25.11.2025).
    \end{minipage}
\end{figure}

Для большей наглядности результат переклассификации представлен на диаграмме Эйлера-Венна (см. Рисунок \ref{fig:Euler}), показывающий пересечение множеств клиентов с дефолтными и недефолтными метками. Красный круг соответствует всем клиентам с исходной недефолтной меткой (96.62\%), зеленый круг -- клиентам с дефолтной меткой (3.38\%), а их пересечение отражает группу аномальных клиентов (32.06\%).
\begin{figure}[H]
\centering
\includegraphics[scale=0.4]{./ch2/graphics/Euler.png}
\caption{Пересечение дефолтных и недефолтных меток с выделением ошибочно недефолтных клиентов}
\label{fig:Euler}
\vspace{10pt}
\begin{minipage}{\textwidth}
        \fontsize{9}{11}\selectfont
        \justifying
        Источник: составлено автором на основе: Соревнование на данных кредитных историй [Электронный ресурс] / Open Data Science. -- URL: \url{https://ods.ai/competitions/dl-fintech-bki} (дата обращения: 25.11.2025).
    \end{minipage}
\end{figure}

Далее необходимо посмотреть как отреагирует алгоритм случайного леса на данных преобразованиях согласно определению \eqref{anom}. Под реакцией алгоритма понимается изменение точности метрик. После выявления аномальных клиентов качество модели значительно улучшилось (см. Таблицу \ref{metrics_after_rfc}). 

\begin{center}
  \begin{longtable}{|c|c|c|c|c|c|c|}
    \caption{Изменение точности метрик}
    \label{metrics_after_rfc} \\
    \hline
    \multicolumn{7}{|c|}{Алгоритм -- Случайный лес}\\
    \hline
    \rowcolor[gray]{.9}
    \specialcell{Flag\\метка} &
    \specialcell{Precision\\точность} &
    \specialcell{Recall\\полнота} &
    \specialcell{f1-score\\f1-мера} &
    \specialcell{Accuracy\\точность}  &
    ROC AUC &
    Количество    \\
    \hline
    \multicolumn{7}{|c|}{Тренировочная выборка}\\
    \hline
    0 & 0.99 & 0.97 & 0.98 &
    \multirow{2}{*}{0.97} &   
    \multirow{2}{*}{0.97} &  
    483921 \\                    
    \cline{1-4} \cline{7-7}
    1 & 0.94 & 0.99 & 0.96 &  
      &      & 266079 \\       
    \hline
    \multicolumn{7}{|c|}{Тестовая выборка} \\
    \hline
    0 & 0.99 & 0.97 & 0.98 &
    \multirow{2}{*}{0.97} &
    \multirow{2}{*}{0.97} &
    161593 \\
    \cline{1-4} \cline{7-7}
    1 & 0.94 & 0.99 & 0.96 & & & 88407 \\
    \hline
  \end{longtable}
  \begin{minipage}{\textwidth}
    \fontsize{9}{11}\selectfont
    \justifying
    Источник: составлено автором на основе: Соревнование на данных кредитных историй [Электронный ресурс] / Open Data Science. -- URL: \url{https://ods.ai/competitions/dl-fintech-bki} (дата обращения: 25.11.2025).
  \end{minipage}
\end{center}

Если сравнить первоначальные результаты, то можно заметить, что если до этого модель игнорировала класс дефолтных клиентов (все метрики равнялись нулю, см. Таблицу \ref{metrics_41}), то после переклассификации метрики точности стали высокими: на обучающей выборке $precision$ = 0.94, $recall$ = 0.99, $f1-score$ = 0.96, аналогичные значения показывает тестовая выборка. При этом можно качество недефолтного класса практически не ухудшилось: $precision$ снизился с 1.00 до 0.99, а метрики $recall$ и $f1-score$ не изменились. Также $ROC\ AUC$ увеличился с 0.51 до 0.97, что свидетельствует о том, что теперь модель хорошо различает дефолтных и недефолтных клиентов. Одновременно увеличилось количество дефолтных клиентов в выборках за счет вновь выявленных аномальных наблюдений, что позволило алгоритму случайного леса показать эффективные результаты на обоих классах. 

\subsection{Статистический тест Колмогорова-Смирнова}

Понятие аномального клиента, изложенное в разделе \ref{subsec:anom}, основано на анализе платежного поведения заемщиков. Применение данного подхода привело к существенному улучшению метрик точности алгоритма случайного леса. Поэтому на следующем шаге проводится статистический анализ, позволяющий увидеть и обосновать, насколько поведение аномальных клиентов отличается от поведения дефолтных клиентов. Для проверки равенства (неравенства) функций распределения двух выборок применяется критерий Колмогорова-Смирнова. А.Н. Колмогоровым\footnote{Колмогоров А.Н. (1903-1987) -- Герой Социалистического Труда, профессор Московского государственного университета, академик АН СССР -- крупнейший математик XX века, является одним из основоположников современной теории вероятности.} и Н.В. Смирновым\footnote{Смирнов Н.В. (1900-1966) -- член-корреспондент АН СССР, один из создателей непараметрических методов математической статистики и теории предельных распределений порядковых статистик.} была доказана следующая теорема:

{\theorem Пусть $F_{X_1}(x)$, $F_{X_2}(x)$ -- выборочные функции распределения
двух независимых выборок объёмами $n$ и $m$. Обозначим:
\[
D_{n,m} = \sup_x \bigl| F_{X_1}(x) - F_{X_2}(x) \bigr|.
\]
}

Тогда для любого $t>0$ выполняется:
\begin{equation}
\forall t>0:\quad
\lim_{n,m\to\infty}
P\!\left(
\sqrt{\frac{nm}{n+m}}\, D_{n,m} \le t
\right)
= K(t)
= \sum_{j=-\infty}^{+\infty} (-1)^j e^{-2 j^2 t^2}.
\label{eq:smirnov_theorem}
\end{equation}

На основе статистики $D_{n,m}$ строится статистика критерия
\[
t_{n,m} = \sqrt{\frac{nm}{n+m}}\, D_{n,m}.
\]

С помощью этой статистики проверяются нулевая и альтернативная гипотезы о совпадении распределений двух выборок:
\[
\begin{cases}
\mathcal{H}_0: F_{X_1}(x) = F_{X_2}(x),\\
\mathcal{H}_1: F_{X_1}(x) \neq F_{X_2}(x).
\end{cases}
\]

При заданном уровне значимости $\alpha$ выбирается критическое значение $K_\alpha$ распределения Колмогорова. Правило принятия решения имеет следующий вид:
\[
\begin{cases}
t_{n,m} \le K_\alpha, & \text{нулевая гипотеза } \mathcal{H}_0 \text{ принимается},\\
t_{n,m} > K_\alpha, & \text{нулевая гипотеза } \mathcal{H}_0 \text{ отвергается в пользу } \mathcal{H}_1.
\end{cases}
\]

Для применения критерия Колмогорова-Смирнова клиенты были разделены на две подвыборки:

\[
\begin{cases}
	X_1 \coloneqq X_{1 \to 1}, \textit{\quad (default)},\\
	X_2 \coloneqq X_{0 \to 1}, \textit{\quad (anom)},
\end{cases}
\] где $X_1$ -- выборка клиентов, являющихся дефолтными до и после условия аномальности;
$X_2$ --  выборка  клиентов, у которых метка изменилась с «0» на «1».

Задача состоит в том, чтобы выяснить, принадлежат ли выборки $X_1$ и $X_2$ одному распределению, применив критерий Колмогорова-Смирнова. Если выборки $X_1$ и $X_2$ принадлежат одному распределению (принятие гипотезы $\mathcal{H}_0$), то платежная дисциплина дефолтных и аномальных клиентов одинакова. Критерий Колмогорова-Смирнова реализуется в листинге \ref{kolmogorov_py} (см. приложение 1). 

Датасет содержит 25 дисциплинарных признаков ($enc\_paym\_0$, \dots, $enc\_paym\_24$), и для каждого из них необходимо применить данный тест, чтобы выяснить, для каких признаков принимается та или иная гипотеза. Поскольку в датасете 3\,000\,000 наблюдений (объемы выборок $n$ и $m$ велики), то при уровне значимости $\alpha$ = 0.05 в качестве критического значения используется $K_\alpha$ = 1.36. 

Анализ показал, что первые 15 месяцев ($enc\_paym\_0$, \dots, $enc\_paym\_14$) поведение аномальных клиентов отличается от поведения дефолтных, а начиная с 16 по 25 месяц ($enc\_paym\_15$, \dots, $enc\_paym\_24$), поведение становится схожим. Так, например, для 15 месяца ($enc\_paym\_14)$ (см. Рисунок \ref{enc_paym_14_def.png}) расчетное значение статистики $t_{n,m}$ = 1.41. Поскольку $K_\alpha$ = 1.36 и $t_{n,m}$ > $K_\alpha$, принимается гипотеза $\mathcal{H}_1$. Это означает, что финансовое поведение аномальных клиентов, раннее относившихся к классу «0», отличается от поведения дефолтных клиентов. 

\begin{figure}[H]
	\centering
	\includegraphics[scale=0.7]{./ch2/graphics/enc_paym_14_def.png}
	\caption{Функции распределения и плотностей распределения дефолтных и аномальных клиентов}
	\label{enc_paym_14_def.png}
	\vspace{10pt}
	\begin{minipage}{\textwidth}
		\fontsize{9}{11}\selectfont
		\justifying
		Источник: составлено автором на основе: Соревнование на данных кредитных историй [Электронный ресурс] / Open Data Science. -- URL: \url{https://ods.ai/competitions/dl-fintech-bki} (дата обращения: 25.11.2025).
	\end{minipage}
\end{figure}

Для 16 месяца $(enc\_paym\_15)$ (см. Рисунок \ref{enc_paym_15_def.png}) значение $t_{n,m}$ = 1.31. В этом случае $t_{n,m}$ < $K_\alpha$, поэтому принимается гипотеза $\mathcal{H}_0$. Распределения значений признака для аномальных и дефолтных клиентов не различаются.

\begin{figure}[H]
	\centering
	\includegraphics[scale=0.7]{./ch2/graphics/enc_paym_15_def.png}
	\caption{Функции распределения и плотностей распределения дефолтных и аномальных клиентов}
	\label{enc_paym_15_def.png}
	\vspace{10pt}
	\begin{minipage}{\textwidth}
		\fontsize{9}{11}\selectfont
		\justifying
		Источник: составлено автором на основе: Соревнование на данных кредитных историй [Электронный ресурс] / Open Data Science. -- URL: \url{https://ods.ai/competitions/dl-fintech-bki} (дата обращения: 25.11.2025).
	\end{minipage}
\end{figure}

Таким образом, поведение аномальных и дефолтных клиентов до 15 месяца различно, а начиная с 16 месяца их поведение становится схожим. Возникает следующий вопрос, если поведение аномальных клиентов отличается от поведения дефолтных ($enc\_paym\_15$, \dots, $enc\_paym\_24$), то не является ли оно более близким к поведению недефолтных клиентов. 

Первые 15 месяцев (см. Рисунок \ref{enc_paym_14_nodef.png}) выплаты аномальных и недефолтных клиентов также существенно различались, поскольку при значении статистики $t_{n,m}$ = 70.92 и условии $t_{n,m}$ > $K_\alpha$ принимается гипотеза $\mathcal{H}_1$, что свидетельствует об их абсолютном различии по поведению.

\begin{figure}[H]
	\centering
	\includegraphics[scale=0.7]{./ch2/graphics/enc_paym_14_nodef.png}
	\caption{Функции распределения и плотностей распределения недефолтных и аномальных клиентов}
	\label{enc_paym_14_nodef.png}
	\vspace{10pt}
	\begin{minipage}{\textwidth}
		\fontsize{9}{11}\selectfont
		\justifying
		Источник: составлено автором на основе: Соревнование на данных кредитных историй [Электронный ресурс] / Open Data Science. -- URL: \url{https://ods.ai/competitions/dl-fintech-bki} (дата обращения: 25.11.2025).
	\end{minipage}
\end{figure}

Если сравнить поведение аномальных и недефолтных клиентов, начиная с 16 месяца (см. Рисунок \ref{enc_paym_15_nodef.png}), то также наблюдается существенное различие. Расчетное значение статистики $t_{n,m}$ = 68.6 и поскольку $t_{n,m}$ > $K_\alpha$, принимается гипотеза $\mathcal{H}_1$.

\begin{figure}[H]
	\centering
	\includegraphics[scale=0.7]{./ch2/graphics/enc_paym_15_nodef.png}
	\caption{Функции распределения и плотностей распределения недефолтных и аномальных клиентов}
	\label{enc_paym_15_nodef.png}
	\vspace{10pt}
	\begin{minipage}{\textwidth}
		\fontsize{9}{11}\selectfont
		\justifying
		Источник: составлено автором на основе: Соревнование на данных кредитных историй [Электронный ресурс] / Open Data Science. -- URL: \url{https://ods.ai/competitions/dl-fintech-bki} (дата обращения: 25.11.2025).
	\end{minipage}
\end{figure}

В результате анализа при сравнении аномальных и дефолтных клиентов была выявлена их схожесть после 16 месяца, тогда как при сравнении аномальных и недефолтных за весь рассматриваемый период наблюдается устойчивое расхождение. Для наглядности результаты сведены в следующую таблицу (см. Таблицу \ref{tab:ks_enc_paym}).

\begin{table}[H]
	\centering
	\caption{Результаты критерия Колмогорова--Смирнова за 25 месяцев}
	\label{tab:ks_enc_paym}
	\begin{tabular}{|c|c|c|c|c|}
		\hline
		\multirow{2}{*}{Месяц} &
		\multicolumn{2}{c|}{Аномальные и дефолтные клиенты} &
		\multicolumn{2}{c|}{Аномальные и недефолтные клиенты} \\
		\cline{2-5}
		& Статистика & \specialcell{Принятие\\гипотезы} & Статистика & \specialcell{Принятие\\гипотезы} \\
		\hline
		\texttt{enc\_paym\_0} & $t_{n,m}$ = 1.65, $t_{n,m}$ > $K_\alpha$ & $\mathcal{H}_1$  & $t_{n,m}$ = 61.40, $t_{n,m}$ > $K_\alpha$ & $\mathcal{H}_1$ \\ \hline
		\texttt{enc\_paym\_1}  & $t_{n,m}$ = 2.49, $t_{n,m}$ > $K_\alpha$ & $\mathcal{H}_1$  & $t_{n,m}$ = 72.44, $t_{n,m}$ > $K_\alpha$ & $\mathcal{H}_1$ \\ \hline
		\texttt{enc\_paym\_2}  & $t_{n,m}$ = 3.09, $t_{n,m}$ > $K_\alpha$ & $\mathcal{H}_1$  & $t_{n,m}$ = 62.21, $t_{n,m}$ > $K_\alpha$ & $\mathcal{H}_1$ \\ \hline
		\texttt{enc\_paym\_3}  & $t_{n,m}$ = 3.33, $t_{n,m}$ > $K_\alpha$ & $\mathcal{H}_1$  & $t_{n,m}$ = 62.03, $t_{n,m}$ > $K_\alpha$ & $\mathcal{H}_1$ \\ \hline
		\texttt{enc\_paym\_4}  & $t_{n,m}$ = 3.29, $t_{n,m}$ > $K_\alpha$ & $\mathcal{H}_1$  & $t_{n,m}$ = 63.85, $t_{n,m}$ > $K_\alpha$ & $\mathcal{H}_1$ \\ \hline
		\texttt{enc\_paym\_5}  & $t_{n,m}$ = 3.20, $t_{n,m}$ > $K_\alpha$ & $\mathcal{H}_1$  & $t_{n,m}$ = 65.22, $t_{n,m}$ > $K_\alpha$ & $\mathcal{H}_1$ \\ \hline
		\texttt{enc\_paym\_6}  & $t_{n,m}$ = 2.88, $t_{n,m}$ > $K_\alpha$ & $\mathcal{H}_1$  & $t_{n,m}$ = 67.04, $t_{n,m}$ > $K_\alpha$ & $\mathcal{H}_1$ \\ \hline
		\texttt{enc\_paym\_7}  & $t_{n,m}$ = 2.50, $t_{n,m}$ > $K_\alpha$ & $\mathcal{H}_1$  & $t_{n,m}$ = 61.43, $t_{n,m}$ > $K_\alpha$ & $\mathcal{H}_1$ \\ \hline
		\texttt{enc\_paym\_8}  & $t_{n,m}$ = 2.28, $t_{n,m}$ > $K_\alpha$ & $\mathcal{H}_1$  & $t_{n,m}$ = 56.28, $t_{n,m}$ > $K_\alpha$ & $\mathcal{H}_1$ \\ \hline
		\texttt{enc\_paym\_9}  & $t_{n,m}$ = 2.00, $t_{n,m}$ > $K_\alpha$ & $\mathcal{H}_1$  & $t_{n,m}$ = 60.88, $t_{n,m}$ > $K_\alpha$ & $\mathcal{H}_1$ \\ \hline
		\texttt{enc\_paym\_10} & $t_{n,m}$ = 1.90, $t_{n,m}$ > $K_\alpha$ & $\mathcal{H}_1$  & $t_{n,m}$ = 68.11, $t_{n,m}$ > $K_\alpha$ & $\mathcal{H}_1$ \\ \hline
		\texttt{enc\_paym\_11} & $t_{n,m}$ = 1.85, $t_{n,m}$ > $K_\alpha$ & $\mathcal{H}_1$  & $t_{n,m}$ = 71.93, $t_{n,m}$ > $K_\alpha$ & $\mathcal{H}_1$ \\ \hline
		\texttt{enc\_paym\_12} & $t_{n,m}$ = 1.72, $t_{n,m}$ > $K_\alpha$ & $\mathcal{H}_1$  & $t_{n,m}$ = 74.84, $t_{n,m}$ > $K_\alpha$ & $\mathcal{H}_1$ \\ \hline
		\texttt{enc\_paym\_13} & $t_{n,m}$ = 1.53, $t_{n,m}$ > $K_\alpha$ & $\mathcal{H}_1$  & $t_{n,m}$ = 73.55, $t_{n,m}$ > $K_\alpha$ & $\mathcal{H}_1$ \\ \hline
		\texttt{enc\_paym\_14} & $t_{n,m}$ = 1.41, $t_{n,m}$ > $K_\alpha$ & $\mathcal{H}_1$  & $t_{n,m}$ = 70.92, $t_{n,m}$ > $K_\alpha$ & $\mathcal{H}_1$ \\ \hline
		\texttt{enc\_paym\_15} & $t_{n,m}$ = 1.31, $t_{n,m}$ < $K_\alpha$ & $\mathcal{H}_0$  & $t_{n,m}$ = 68.86, $t_{n,m}$ > $K_\alpha$ & $\mathcal{H}_1$ \\ \hline
		\texttt{enc\_paym\_16} & $t_{n,m}$ = 1.17, $t_{n,m}$ < $K_\alpha$ & $\mathcal{H}_0$  & $t_{n,m}$ = 68.26, $t_{n,m}$ > $K_\alpha$ & $\mathcal{H}_1$ \\ \hline
		\texttt{enc\_paym\_17} & $t_{n,m}$ = 1.03, $t_{n,m}$ < $K_\alpha$ & $\mathcal{H}_0$  & $t_{n,m}$ = 67.03, $t_{n,m}$ > $K_\alpha$ & $\mathcal{H}_1$ \\ \hline
		\texttt{enc\_paym\_18} & $t_{n,m}$ = 0.99, $t_{n,m}$ < $K_\alpha$ & $\mathcal{H}_0$  & $t_{n,m}$ = 65.48, $t_{n,m}$ > $K_\alpha$ & $\mathcal{H}_1$ \\ \hline
		\texttt{enc\_paym\_19} & $t_{n,m}$ = 0.84, $t_{n,m}$ < $K_\alpha$ & $\mathcal{H}_0$  & $t_{n,m}$ = 64.54, $t_{n,m}$ > $K_\alpha$ & $\mathcal{H}_1$ \\ \hline
		\texttt{enc\_paym\_20} & $t_{n,m}$ = 0.79, $t_{n,m}$ < $K_\alpha$ & $\mathcal{H}_0$  & $t_{n,m}$ = 62.74, $t_{n,m}$ > $K_\alpha$ & $\mathcal{H}_1$ \\ \hline
		\texttt{enc\_paym\_21} & $t_{n,m}$ = 0.80, $t_{n,m}$ < $K_\alpha$ & $\mathcal{H}_0$  & $t_{n,m}$ = 60.50, $t_{n,m}$ > $K_\alpha$ & $\mathcal{H}_1$ \\ \hline
		\texttt{enc\_paym\_22} & $t_{n,m}$ = 0.80, $t_{n,m}$ < $K_\alpha$ & $\mathcal{H}_0$  & $t_{n,m}$ = 58.80, $t_{n,m}$ > $K_\alpha$ & $\mathcal{H}_1$ \\ \hline
		\texttt{enc\_paym\_23} & $t_{n,m}$ = 0.80, $t_{n,m}$ < $K_\alpha$ & $\mathcal{H}_0$  & $t_{n,m}$ = 57.53, $t_{n,m}$ > $K_\alpha$ & $\mathcal{H}_1$ \\ \hline
		\texttt{enc\_paym\_24} & $t_{n,m}$ = 0.68, $t_{n,m}$ < $K_\alpha$ & $\mathcal{H}_0$  & $t_{n,m}$ = 53.10, $t_{n,m}$ > $K_\alpha$ & $\mathcal{H}_1$ \\ \hline
	\end{tabular}
	
	\vspace{3mm} 
	
	\begin{minipage}{0.98\textwidth}
		\fontsize{9}{11}\selectfont
		\justifying
		Примечание: $K_\alpha$ = 1.36 при уровне значимости $\alpha$ = 0.05.
	\end{minipage}
\end{table}

\section{Базовые алгоритмы оценки вероятности возврата кредита}
\subsection{Деревья решений}
\subsection{Случайный лес}
\subsection{Градиентный бустинг}
\subsection{Нейронные сети}
