\section*{\centering Введение} %\addcontentsline{toc}{section}{Введение}
\addcontentsline{toc}{chapter}{Введение}
	
	\textbf{Актуальность темы исследования}. В настоящее время, в условиях технологического развития, искусственный интеллект (далее ИИ) стал необходимым инструментом, позволяющим с высокой скоростью обрабатывать большие массивы данных. Для банковского сектора управление кредитными рисками является одной из ключевых задач, в частности оценка кредитоспособности заемщиков. В современном финансовом секторе Казахстана использование кредитного скоринга поднимает важный вопрос о справедливом доступе к потребительскому кредитованию. Кредитный скоринг -- это статистический процесс, позволяющий прогнозировать вероятность дефолта заемщика банка. Значимость кредитного скоринга для Казахстана определяется его ролью в снижении долговой нагрузки населения, ограничении массового заимствования и обеспечении устойчивого и эффективного развития финансового сектора.
	
	С развитием услуг микрокредитования и предоставления кредитов в рассрочку банки стали получать значительно больше заявок на кредит, что на первоначальном этапе было выгодно финансовым учреждениям. Однако по мере роста количества заявок анализ разнообразных кредитных историй клиентов стал сложной задачей для сотрудников банка, что приводило к снижению прибыли. В связи с этим возникает необходимость в автоматизации и оптимизации процесса оценки и выдачи кредитов с помощью технологий ИИ, что позволяет минимизировать кредитные риски для обеих сторон. Практическая значимость работы заключается в том, что результаты исследования могут быть использованы финансовыми организациями для повышения эффективности принятия кредитных решений. Применение алгоритмов машинного и глубокого обучения позволит банкам принимать более точные и обоснованные решения о кредитоспособности клиентов, снижая вероятность возникновения финансовых рисков.
	
	\textbf{Степень разработанности темы исследования}. Применение искусственного интеллекта в банковском кредитовании рассматривались в работах H. Sadok, F. Sakka и M.E. El Maknouzi, в магистерской диссертации Ш. Сяоюй, а также в исследованиях Г.З. Зиятбековой, А.А. Давыдовой и О.Л. Ксенофонтовой, посвященных использованию методов машинного обучения и интеллектуального анализа данных в банковской сфере. Данные работы свидетельствуют об устойчивом научном интересе внедрения ИИ в деятельность финансовых учреждений. Тем не менее для Казахстана данное направление является новым в контексте цифровизации финансового сектора и, следовательно, требует дальнейшего исследования.
	
	\textbf{Цель и задачи исследования} -- разработать метод прогнозирования дефолта клиента банка при выдаче кредита на основе методов машинного и глубокого обучения.
	Для достижения поставленной цели были определены следующие задачи:
	\begin{enumerate}
	       \item Провести анализ и предобработку данных о заемщиках.
               \item Обучить и протестировать модели классификации для прогнозирования дефолта.
               \item Подобрать гиперпараметры моделей для повышения качества прогнозирования дефолтных клиентов.
               \item Оценить качество моделей и проанализировать полученные результаты.
        \end{enumerate}
        
        \textbf{Объектом исследования} является выданный кредит банком и связанное с ним наступление либо отсутствие дефолта по выданным кредитам.
       
        \textbf{Предметом исследования} является кредитный скоринг для оценки риска невозврата кредита.
        
        Для достижения целей и задач исследования использовались следующие методы:
        \begin{itemize}
               \item машинного и глубокого обучения;
               \item статистического анализа;
               \item оценивания моделей;
               \item визуализации исторических данных.
        \end{itemize}
        
        \textbf{Научная новизна исследования} исследования заключается в следующем:
         \begin{itemize}
               \item после завершения работы добавим
        \end{itemize}
        
        \textbf{Информационной базой исследования} являлись платформы: kaggle, github, stackoverflow, нормативно-правовая база, научные статьи и монографии.
        
        \textbf{Структура и объем работы}. Выпускная квалификационная работа состоит из 3 глав, заключения, списка литературы, Х таблиц, Y рисунков и Z приложений.
        
        \textbf{Объем исследовательской работы:} 50 страниц.